\documentclass[journal,twocolumn]{IEEEtran}

\usepackage[utf8]{inputenc}
\usepackage{graphicx}
\usepackage{amssymb}
\usepackage{mathtools}
\usepackage{amsmath}
\providecommand{\pr}[1]{\ensuremath{\Pr\left(#1\right)}}
\providecommand{\cbrak}[1]{\ensuremath{\left\{#1\right\}}}

\title{Assignment 9}
\author{Gollapudi Sasank CS21BTECH11019}

\begin{document}
\maketitle
\section*{Question : }
Players $X$ and $Y$ roll dice alternately starting with $X$. The player that rolls $11$ wins. Show that the probability $p$ that $X$ wins equals $18/35$.
\section*{Solution : }
Let the Random Variables $A,B$ denote the following : \\
$A=0$ : The person who starts the game  wins \\
$A=1$ : The person who starts the game  loses \\
$B=0$ : $11$ occurs at the first throw \\
$B=1$ : $11$ does not occur at the first throw \\
Given,
\begin{align}
\pr{A=0} &= p \\
\Rightarrow \pr{A=1} &= 1-p \\
\pr{B=0} &= \frac{2}{36} = \frac{1}{18} \\
\Rightarrow \pr{B=1} &= \frac{34}{36} = \frac{17}{18}
\end{align}
The Events $B=0$ and $B=1$ form a partition to the Sample Space. \\
\begin{align}
\Rightarrow \pr{A=0} &= \pr{(A=0)((B=0)+(B=1))} \\
\Rightarrow \pr{A=0} &= \pr{(A=0)(B=0)+(A=0)(B=1)}
\end{align}
The Events $(A=0)(B=0)$ and $(A=0)(B=1)$ are mutually exclusive 
\begin{align}
\Rightarrow \pr{A=0} &= \pr{(A=0)(B=0)} + \pr{(A=0)(B=1)} \\
\Rightarrow \pr{A=0} &= \pr{A=0|B=0}\pr{B=0} + \pr{A=0|B=1}\pr{B=1}
\end{align}
$\pr{A=0|B=0} = 1$ because  if 11 occurs at first throw $X$ wins.\\
Now the event $(A=0|B=1)$ is the case where $X$ wins when 11 does not occur at first throw. So in this case if we consider the game from the second throw then $Y$ throws first. But here we need the probability of the case where the person who starts the game loses i.e $\pr{A=1} = 1-p $.\\
$\therefore \pr{A=0|B=1} = 1-p $
\begin{align}
\Rightarrow p &= 1 \times \frac{1}{18} + (1-p) \times \frac{17}{18}\\
\Rightarrow p &= \frac{1 + 17 - 17p}{18} \\
\Rightarrow 18p &= 18 - 17p \\
\Rightarrow 35p &= 18 \\
\Rightarrow p &= \frac{18}{35}
\end{align}
\end{document}